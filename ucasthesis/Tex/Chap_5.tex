\chapter{结论与展望}\label{chap:future}
%icache目前还是直接映射的。
%被诟病的微结构
%截至中期,工程实现依旧采用的是直接映射的存储结构,后期会将其调整为4路组相连结构。这里有两个目的。其一,risc-v对于页表的规定固定为4KB,为了做到cpu的tlb虚实转化和访问cache的并行,所以cache core的每一路的容量都至多是4KB,也就是说用虚实地址共享的地址(不用tlb cam来进行虚实转换)来索引cache core。但是4KB的容量显然不够的,但是每一路至多4KB,当然可以用硬件来解决顶着色对每一路的cache core进行扩容,但是这就很复杂了。所以另一种思路就是增加路数,比如四路;其二,四路比一路cache的替换率要低,这不光是容量的问题,四路的每路4KB的cache在一般情况下要比容量相同的16KB的单路cache的替换率都低。

%所以对于cache core最终的组织形式目前的想法是4路组相连,每一路4KB,采用LRU或者随机替换或者一种兼而有之的方法。不会参考alpha21464一样做路预测,一是为了降低复杂度,二是因为估计在自己的设计中延迟时序允许。
%基于CAM的重命名表
%更侧重微结构的设计
	通过章节\ref{chap:analysis}的分析,相比于单发射五级静态流水线再到双发射五级静态流水线,双发射乱序处理器在执行程序的效率上有了比较明显的提升。但是另一方面,也可以看出双发射乱序处理器对于转移猜测的预测率是非常敏感的,一些测试程序如coremark和meidian由于目前实现的转移预测正确率不到80\%,而导致在没有访存延迟的情况下,反而比双发射五级静态流水IPC要差。所以当务之急是通过牺牲面积的方式引入效果更好的转移预测方式,提高转移预测率从而提高双发射乱序执行程序的效率。这也是前端的首要改进工作。对于中间层,在最近的一段时间内,通过仔细的思考,发现基于CAM的重命名表会比基于RAM的重命名表面积更小,时序更好,而且逻辑同样的简单。对于后端,首要的任务是加入非阻塞式的dcache,包括对后端时序更为细致的打磨。
	
	所以对于眼前的展望就是将这个超标量宽度为2的双发射乱序处理器的效率继续提升。对于稍远的计划,还是要把整个系统做大做完整,比如做成64位架构的,然后加入浮点的逻辑,使之能够成为像BOOM一样具有实用价值的处理器核。而且光从流水线的划分粗略的分析,在同为前端超标量宽度为2的设置下,后端设计的调度策略和机制是比BOOMv2更好,时序上粗略估计也不必BOOM差。所以在做出完整系统的同时,性能上的目标是要超过BOOMv2,基本上要做到ARM Cortex A57的水平。更长远的展望是,因为国内的大方向还是以MIPS体系结构为主,所以肯定会将目前基于RISC-V的设计改版成MIPS的版本,然后通过不懈的努力,使自己的设计像BOOM一样能够有机会流片,在实际的应用中得到检验。
	
	
	